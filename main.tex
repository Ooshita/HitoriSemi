\documentclass{article}
\documentclass[dvipdfmx]{jsarticle}
\usepackage[utf8]{inputenc}
\usepackage[ipaex]{pxchfon}

\title{DensityRatioEstimation}
\author{大下 範晃}
\date{February 2019}

\begin{document}

\maketitle

\section{イントロダクション}

$\mathcal{X} \subset \mathbf{R}^d $ はデータドメインとする
\begin{equation}
    \left\{ x _ { i } ^ { \mathrm { nu } } \right\} _ { i = 1 } ^ { n _ { \mathrm { nu } } } \stackrel { \mathrm { i.i.d. } } { \sim } p _ { \mathrm { nu } } ^ { * } ( \boldsymbol { x } ) \quad and \quad \left\{ x _ { j } ^ { \mathrm { de } } \right\} _ { j = 1 } ^ { n _ { \mathrm { de } } } \stackrel { \mathrm { i.i.d. } } { \sim } p _ { \mathrm { de } } ^ { * } ( \boldsymbol { x } )
\end{equation}

ドメイン$\mathcal{X}$に対して$p(x)^*$は完全に正と仮定する.\\
つまり,$p(x)^* > 0, x \in \mathcal{X}$

ゴールは以下に示す密度比を推定すること\\
\begin{equation}
    r(x)^* = \frac{p(x)^*_{nu}}{p(x)^*_{de}}
\end{equation}

\subsection{基本的枠組み}

次のサンプルを仮定する.
\begin{equation}
    \left\{ \boldsymbol { x } _ { k } \right\} _ { k = 1 } ^ { n } \stackrel { \mathrm { i.i.d. } } { \sim } p ^ { * } ( \boldsymbol { x } )
\end{equation}

密度推定のゴールは$\{x_k\}_{k=1}^n$からの真の密度$p(x)^*$の推定量$\hat{p(x)}$を取得すること.\\

簡易的な方法\\
\begin{equation}
    \hat{r} = \frac{\hat{p(x)_{nu}}}{\hat{p(x)_{de}}}
\end{equation}
として求める.

\begin{thebibliography}{99}
\item
DensityRatioEstimation in Machine Learning
Masashi Sugiyama Taiji Suzuki Takafumi Kanamori
\end{thebibliography}
\end{document}
